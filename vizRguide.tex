\documentclass[a4paper,twoside]{book}
\let\newfloat\undefined
\usepackage[framemethod=1,skipbelow=\topskip,skipabove=\topskip]{mdframed}
\usepackage{graphicx}
\usepackage{url}
\usepackage{lipsum}
\usepackage{xspace}
\usepackage{fancyhdr}
\usepackage[Sonny]{fncychap}

\usepackage{caption}
\usepackage{floatrow}
\floatsetup[figure]{facing=yes,capbesideposition={top,inside},capbesidesep=quad}
\setlength{\floatsep}{1.25\baselineskip plus 3pt minus 2pt}
\setlength{\intextsep}{1.2ex}

\usepackage{makeidx}


\usepackage{fontspec,xltxtra,xunicode}
\defaultfontfeatures{Scale=MatchLowercase}
\setromanfont[Mapping=tex-text]{Minion Pro} 
\setsansfont[Mapping=tex-text]{Myriad Pro} 
\setmonofont[Scale=0.91]{Inconsolata}
\newfontfamily{\titlefont}[Scale=1.4]{Fertigo Pro}
\definecolor{grey30}{rgb}{.3,.3,.3}
\definecolor{grey70}{rgb}{.7,.7,.7}
\definecolor{mybrown}{rgb}{.5,.4,.3}
\definecolor{myred}{rgb}{.75,.19,.19}
\definecolor{myred2}{rgb}{.87,.38,.38}
\definecolor{myred3}{rgb}{.65,.38,.38}
\definecolor{myred4}{rgb}{.83,.64,.64}
\definecolor{shadecolor}{rgb}{.87,.93,.89}

\usepackage[formats]{listings}
\lstdefineformat{tilde}{\~=\textcolor{myred3}{$\mathtt{\sim}$}}

\lstloadlanguages{R} 
\lstdefinelanguage{Renhanced}[]{R}{% 
   morekeywords={xyplot}, 
   deletekeywords={*}}
\lstset{language=Renhanced,
	format=tilde,
	keepspaces=true,
	%escapeinside={\%*}{*)},
	extendedchars=true, 
	alsoletter={.},
	upquote=true,
	basicstyle=\small\ttfamily, 
   	commentstyle=\color{myred4}, 
   	keywordstyle=\color{myred3}, 
   	showstringspaces=false, 
   	index=[1][keywords], 
   	indexstyle=\indexfoo} 

\newcommand{\indexfoo}[1]{\index{#1@{\tt #1}}} 
\makeatletter
\lst@AddToHook{TextStyle}{\let\lst@basicstyle\normalsize\ttfamily}
\makeatother
\renewcommand{\texttt}[1]{\lstinline{#1}}
\newcommand{\R}{\textsf{R}\xspace}

\mdfsetup{innerbottommargin=0pt,leftmargin=0pt,rightmargin=0pt,%
          backgroundcolor=shadecolor,linecolor=none,roundcorner=5}
\BeforeBeginEnvironment{lstlisting}{\begin{mdframed}\vskip-.5\baselineskip}
\AfterEndEnvironment{lstlisting}{\end{mdframed}}

% I'm using temporarily this example title page, roughly copy/pasted from the 
% Memoir class package.
\definecolor{Medium}{gray}{.6}
\newlength{\tpheight}\setlength{\tpheight}{0.9\textheight}
\newlength{\txtheight}\setlength{\txtheight}{0.9\tpheight}



\makeindex

\begin{document}
\pagenumbering{arabic}
\thispagestyle{empty}
\begingroup% T&H Typography
\raggedleft
\vspace*{\baselineskip}
{\Large Christophe Lalanne}\\[0.167\txtheight]
{\Large\titlefont A Visual Guide to}\\[\baselineskip]
{\textcolor{Medium}{\Huge\titlefont R Graphics}}\\[\baselineskip]
{\tiny\titlefont With 123 illustrations}\par
\vfill
{\includegraphics[scale=.5]{logo}}\par
\vspace*{3\baselineskip}
\endgroup

\frontmatter
\tableofcontents
\mainmatter
\chapter{Introduction}

\chapter{Data management}

The most common structure to store a data set with mixed-type
variables is a \texttt{data.frame}.

\lipsum

\chapter{Two-way graphics}




\section{Scatterplots}
The basic \R command for displaying a two-way scatterplot is
\texttt{xyplot}. A command like \texttt{xyplot(y ~ x)} will produce a
2D plot almost identical to what would be obtained using base
graphics, \texttt{plot(x, y)}. However, the default layout is
generally better and it looks more pretty.

\begin{figure}[H]
\begin{lstlisting}
xyplot(dist ~ speed, data=cars)
\end{lstlisting}
\fcapside[\FBwidth]
{\includegraphics[width=2.5in]{fig_001-crop}}
{\caption*{This basic scatterplot show default options when calling the
\texttt{xyplot} command. The formula interface is used to plot
\texttt{dist} ($y$-axis) as a function of \texttt{speed} ($x$-axis),
with automatic determination of axis units.}}
\end{figure}



\begin{figure}[H]
\begin{lstlisting}
xyplot(dist ~ speed, data=cars, pch=rbinom(nrow(cars), 1, .5)+1)
\end{lstlisting}
\fcapside[\FBwidth]
{\includegraphics[width=2.5in]{fig_002-crop}}
{\caption*{We pick a random symbol () for each observation,
using the \texttt{pch=} argument. In fact, this argument is
transferred to the \texttt{panel.xyplot} function that acts as the
default panel function.}}
\end{figure}


\begin{figure}[H]
\begin{lstlisting}
xyplot(dist ~ speed, data=cars, 
       col=with(cars, cut(speed, breaks=quantile(speed), 
                          include.lowest=TRUE)))
\end{lstlisting}
\fcapside[\FBwidth]
{\includegraphics[width=2.5in]{fig_003-crop}}
{\caption*{Color (\texttt{col=}) of each observation depends of the quartile they
    belong to.}}
\end{figure}

\begin{figure}[H]
\begin{lstlisting}
xyplot(dist ~ speed, data=cars, aspect=.3, type=c("p", "r"),
             col.line="grey75",
             panel=function(x, y, ...) {
               panel.xyplot(x, y, ...)
               panel.rug(x, y, ...)
             })
\end{lstlisting}
\fcapside[\FBwidth]
{\includegraphics[width=2.5in]{fig_004-crop}}
{\caption*{We pick a random color--black (1) or red (2)--for each observation,
using the \texttt{col=} argument. In fact, this argument is
transferred to the \texttt{panel.xyplot} function that acts as the
default panel function.}}
\end{figure}

\lipsum[1]

\chapter{Multi-way graphics}



\appendix
\backmatter

\printindex

\end{document}
