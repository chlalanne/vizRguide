\documentclass[a4paper,twoside]{book}
\let\newfloat\undefined
\usepackage[framemethod=1,skipbelow=\topskip,skipabove=\topskip]{mdframed}
\usepackage{graphicx}
\usepackage{url}
\usepackage{lipsum}
\usepackage{xspace}
\usepackage{fancyhdr}
\usepackage[Sonny]{fncychap}
\usepackage{marginnote}
\usepackage{caption}
\usepackage{floatrow}
\floatsetup[figure]{facing=yes,capbesideposition={top,inside},capbesidesep=quad}
\setlength{\floatsep}{1.25\baselineskip plus 3pt minus 2pt}
\setlength{\intextsep}{1.2ex}
\usepackage{wasysym}
\usepackage{makeidx}
\usepackage{pdfpages}
\newcounter{fig}
\setcounter{fig}{1}
\newcommand{\img}[1]{\includegraphics[page={\value{fig}},width=2in]{#1}\stepcounter{fig}}


\immediate\write18{sh ./vc}
\input{vc}

\usepackage[style=verbose-trad2,natbib=true]{biblatex}
\bibliography{refs}
\usepackage[hang,flushmargin]{footmisc} 

\usepackage{fontspec,xltxtra,xunicode}
\defaultfontfeatures{Scale=MatchLowercase}
\setromanfont[Mapping=tex-text]{Minion Pro} 
\setsansfont[Mapping=tex-text]{Myriad Pro} 
\setmonofont[Scale=0.91]{Inconsolata}
\newfontfamily{\titlefont}[Scale=1.4]{Fertigo Pro}
\definecolor{grey30}{rgb}{.3,.3,.3}
\definecolor{grey70}{rgb}{.7,.7,.7}
\definecolor{mybrown}{rgb}{.5,.4,.3}
\definecolor{myred}{rgb}{.75,.19,.19}
\definecolor{myred2}{rgb}{.87,.38,.38}
\definecolor{myred3}{rgb}{.65,.38,.38}
\definecolor{myred4}{rgb}{.83,.64,.64}
\definecolor{shadecolor}{rgb}{.87,.93,.89}

\usepackage[formats]{listings}
\lstdefineformat{tilde}{\~=\textcolor{myred3}{$\mathtt{\sim}$}}

\lstloadlanguages{R} 
\lstdefinelanguage{Renhanced}[]{R}{% 
   morekeywords={data.frame,within,as.numeric,cut,xyplot,histogram,%
                 groups,breaks,include.lowest,with,pch,panel.xyplot,%
                 replicate,alpha,colorRampPalette,jitter.x,amount,%
                 panel.rug,brewer.pal,rgb,jitter.y,Hmisc::describe,%
                 Hmisc::summary.formula,type,scales,xscale.components,%
                 yscale.components,xscale.components.log10.3,%
                 yscale.components.log10.3,box.ratio,limits,at,%
                 panel.bwplot,panel.points,cex,zoo,plot.points,%
                 ref,panel.histogram,panel.mathdensity,dmath,border,%
                 densityplot,nint,tck,alternating,boxplot.stats,%
                 na.rm}, 
   deletekeywords={*}}
\lstset{language=Renhanced,
	format=tilde,
	keepspaces=true,
	%escapeinside={\%*}{*)},
	extendedchars=true, 
	alsoletter={.},
	upquote=true,
	basicstyle=\small\ttfamily, 
   	commentstyle=\color{myred4}, 
   	keywordstyle=\color{myred3}, 
   	showstringspaces=false, 
   	index=[1][keywords], 
   	indexstyle=\indexfoo} 

\newcommand{\indexfoo}[1]{\index{#1@\textcolor{grey30}{\tt #1}}} 
\makeatletter
\lst@AddToHook{TextStyle}{\let\lst@basicstyle\normalsize\ttfamily}
\makeatother
\renewcommand{\texttt}[1]{\lstinline{#1}}
\newcommand{\R}{\textsf{R}\xspace}
\newcommand{\mytilde}{\textcolor{myred3}{$\sim$}\xspace}

\mdfsetup{innerbottommargin=0pt,leftmargin=0pt,rightmargin=0pt,%
          backgroundcolor=shadecolor,linecolor=none,roundcorner=5}
\BeforeBeginEnvironment{lstlisting}{\begin{mdframed}\vskip-.5\baselineskip}
\AfterEndEnvironment{lstlisting}{\end{mdframed}}

% I'm using temporarily this example title page, roughly copy/pasted from the 
% Memoir class package.
\definecolor{Medium}{gray}{.6}
\newlength{\tpheight}\setlength{\tpheight}{0.9\textheight}
\newlength{\txtheight}\setlength{\txtheight}{0.9\tpheight}
\renewcommand*{\marginfont}{\color{grey70}\sffamily\footnotesize}


\setlength{\parindent}{0pt}
\setlength{\parskip}{1ex plus 0.5ex minus 0.2ex}
\makeindex

\begin{document}
\pagenumbering{arabic}
\thispagestyle{empty}
\begingroup% T&H Typography
\raggedleft
\vspace*{\baselineskip}
{\Large\textsf{Christophe Lalanne}}\\[0.167\txtheight]
{\Large\titlefont A Visual Guide to}\\[\baselineskip]
{\textcolor{Medium}{\Huge\titlefont R Graphics}}\\[\baselineskip]
{\tiny\titlefont With 123 illustrations}\par
\vfill
{\includegraphics[scale=.5]{logo}\\{\small \tt \GITAbrHash}\\{\small \tt \VCDateTEX}}\par
\vspace*{3\baselineskip}
\endgroup

\frontmatter
\tableofcontents
\mainmatter
\chapter{Introduction}

\chapter{Data management}

\section{Structuring data}
In R, the most common structure used to store a data set with
mixed-type variables is a \texttt{data.frame}. Such an \R object
presents several characteristics that makes it most appropriate for
managing statistical data structure, with few exceptions (e.g., when
one only has to work with aggregated data or two-way tables). It
should be noted that other data structures might be more appropriate,
for example when one is interested in time series analysis, but see
the \texttt{zoo} package\autocite{zeilis05}.

Many \R functions accept \texttt{data.frame} as input, and further
allow to subset or index it for computation or visualization
purpose. In addition to receiving a \texttt{data.frame}, some \R
commands allow to use a \emph{formula} notation, where the right and
left-hand side are separated by the \mytilde (tilde) operator. The
use of \index{formula} together with a \texttt{data.frame} simplify the
accession of variable in a given environment.
This is especially true when using the \texttt{lattice} package which
is entirely based on formula, even if this is not apparent at
first sight.

\section{Managing data}
Consider, for example, the \emph{low birth study} which is discussed
at length in Hosmer and Lemeshow's textbook on logistic regression
\autocite{hosmer89}. A quick look at the variables should make it
clear that they won't be treated the way we like them to be
considered: mother's ethnicity status (\texttt{race}) takes three
integer values, without any explicit meaning.

\begin{verbatim}
data(birthwt, package="MASS")
str(birthwt)
\end{verbatim}

Instead, we might want to recode all categorical predictors as \R
factors, like this:
\begin{verbatim}
birthwt <- within(birthwt, {
  low <- factor(low, labels=c("No","Yes"))
  race <- factor(race, labels=c("White","Black","Other"))
  smoke <- factor(smoke, labels=c("No","Yes"))
  ui <- factor(ui, labels=c("No","Yes"))
  ht <- factor(ht, labels=c("No","Yes"))
})
\end{verbatim}

In case we would like to consider one of the above factors as a
numerical variable, we can now use \texttt{as.numeric} and \R will
take care of attributing the lowest integer score to the baseline
category. Of course, there might be occasion where we would like to
change that reference level; or, we might want to collapse two
discrete categories. Again, there are simple commands to do that, for
example:
\begin{verbatim}
birthwt$low <- relevel(birthwt$low, "Yes")
levels(birthwt$race)[2:3] <- "Black+Other"
\end{verbatim}

Another common task consists in transforming some predictors, either
for visualization purpose or when building an explantory or predictive
model. As a simple example, we can imagine centering some of the
predictors of interest, like \texttt{age}, in the above example. The
\texttt{within} or \texttt{transform} command can be used to append
the centered variable to the list of variables present in the
\texttt{data.frame}:
\begin{verbatim}
birthwt <- transform(birthwt, age.c=scale(age, scale=FALSE))
\end{verbatim}

Likewise, we may want to recode previous premature labours
(\texttt{ptl}) as yes/no and number of physician visits during the
first trimester (\texttt{ftv}) as one/more than one, like shown below
(we show two different syntax that basically perform the same task by
relying on \R indexing):
\begin{verbatim}
birthwt <- transform(birthwt, ptl.yn=factor(ptl > 0, labels=c("No","Yes")), 
                     ftv.c=factor(ifelse(ftv < 2, "1", "2+")))
\end{verbatim}

If there is some reason to treat \texttt{ftv} as an ordered factor,
a command like
\begin{verbatim}
as.ordered(Hmisc::cut2(birthwt$ftv, c(0, 1, 2, 6)))
\end{verbatim}
might do the job. (This would also be possible with the base
\texttt{cut} function, but the one in
\texttt{Hmisc}\autocite{harrell11} has better default options and it
is more flexible.)

\section{Indexing, subsetting, conditioning}
A lot of statistical operations that practictioners use to apply on a
given dataset are mostly variations around the idea of indexing or
subsetting. 

We have already seen an example of indexing when we recoded factor
levels of two explanatory variables in the low birth dataset.

\section{Summarizing data}
Statisticians generally spend a great part of their time in data
cleansing, data transformation or re-expression\autocite{hoaglin83},
and data visualization. There are numerous \R functions that will
facilitate the task of data checking (\texttt{Hmisc::describe}
provides ``codebook'' facilities), summarizing data (\texttt{summary},
\texttt{Hmisc::summary.formula}), aggregating data (\texttt{tapply},
\texttt{aggregate}). 

The \texttt{plyr} package\autocite{wickham11} offers a general
solution to those kind of tasks.

\chapter{Univariate distributions}

\section{Stripchart}


stripplot(~ rnorm(200), jitter.data = TRUE, factor = 0.8)
stripplot(rnorm(30) ~ 1, horizontal=F,
          scales=list(x=list(draw=FALSE)), xlab="")

\section{Histograms}
\begin{figure}[H]
\begin{lstlisting}
histogram(~ waiting, data=faithful)
\end{lstlisting}
  \fcapside[\FBwidth] {\img{figs_lattice-crop}}
  {\caption*{\marginnote{\textbf{faithful}. Waiting time between
        eruptions and the duration of the eruption for the Old
        Faithful geyser in Yellowstone National Park, Wyoming, USA.}A
      simple histogram of waiting time expressed as density. Note that
      forgetting the \mytilde operator will raise an error message.}}
\end{figure}

\begin{figure}[H]
\begin{lstlisting}
histogram(~ waiting, data=faithful, type="count")
\end{lstlisting}
  \fcapside[\FBwidth] {\img{figs_lattice-crop}}
  {\caption*{Same as above but with counts instead of density.}}
\end{figure}

\begin{figure}[H]
\begin{lstlisting}
histogram(~ waiting, data=faithful, border=NA, nint=12)
\end{lstlisting}
  \fcapside[\FBwidth] {\img{figs_lattice-crop}}
  {\caption*{Using a finer grid size and removing the borders. (To
      remove all colors, use \texttt{col=NULL}.)}}
\end{figure}

\begin{figure}[H]
\begin{lstlisting}
x <- rnorm(100, mean=12, sd=2)
histogram(~ x, type="density", border=NA,
          panel=function(x, ...) {
            panel.histogram(x, ...)
            panel.mathdensity(dmath=dnorm, col="#BF3030",
                              args=list(mean=mean(x),sd=sd(x)))
          })
\end{lstlisting}
  \fcapside[\FBwidth] {\img{figs_lattice-crop}}
  {\caption*{An example where we superimposed a normal density with
      parameters estimated from the sample.)}}
\end{figure}

\section{Density plots}

\begin{figure}[H]
\begin{lstlisting}
densityplot(~ waiting, data=faithful)
\end{lstlisting}
  \fcapside[\FBwidth] {\img{figs_lattice-crop}}
  {\caption*{}}
\end{figure}

\begin{figure}[H]
\begin{lstlisting}
densityplot(~ waiting, data=faithful, plot.points="rug")
\end{lstlisting}
  \fcapside[\FBwidth] {\img{figs_lattice-crop}}
  {\caption*{}}
\end{figure}

\begin{figure}[H]
\begin{lstlisting}
densityplot(~ waiting, data=faithful, plot.points=FALSE, ref=TRUE)
\end{lstlisting}
  \fcapside[\FBwidth] {\img{figs_lattice-crop}}
  {\caption*{}}
\end{figure}


\section{Barcharts}

\section{Dotcharts}

\section{Quantile plots}

\section{Boxplots}

\begin{figure}[H]
\begin{lstlisting}
bwplot(~ height, data=women)
\end{lstlisting}
  \fcapside[\FBwidth] {\img{figs_lattice-crop}}
  {\caption*{\marginnote{\textbf{women}. This data set gives the
        average heights and weights for American women aged 30-39.}
      Boxplot are shown in horizontal mode by default. Internally, the
      \texttt{boxplot.stats} function is used so that hinges
      corresponds to the first and third quartile while notches
      extends to $\pm 1.58\textrm{IQR}/\sqrt{n}$. It provides a visual
      summary analogous to Tukey's five number (see
      \texttt{fivenum}).}}
\end{figure}

\begin{figure}[H]
\begin{lstlisting}
bwplot(~ height, data=women, box.ratio=.5)
\end{lstlisting}
  \fcapside[\FBwidth] {\img{figs_lattice-crop}} {\caption*{This is the
      same picture but with a thinner box. When there are several
      boxplots to draw side by side, this might be useful.}}
\end{figure}

\begin{figure}[H]
\begin{lstlisting}
bwplot(~ height, data=women, box.ratio=.5,
       scales=list(x=list(limits=c(50, 80), at=seq(50, 80, by=2))))
\end{lstlisting}
  \fcapside[\FBwidth] {\img{figs_lattice-crop}} {\caption*{A more
      detailed $x$-scale has been created (without prejudice to its
      usefulness) by simply updating the \texttt{scales=} argument.}}
\end{figure}

\begin{figure}[H]
\begin{lstlisting}
bwplot(~ height, data=women, aspect=.5, 
       panel=function(x, ...) {
         panel.bwplot(x, pch="|", ...)
         panel.points(mean(x), 1, pch=19, cex=1)
       })
\end{lstlisting}
  \fcapside[\FBwidth] {\img{figs_lattice-crop}} {\caption*{It is
      possible to alter the way boxplot are drawn, but also the
      statistical summaries that are displayed. For instance, in the
      above code, we computed the mean (drawn as a vertical bar inside
      the box) in addition to the median. Of note, if there are
      missing values, we should add \texttt{na.rm=TRUE} when calling
      \texttt{mean(x)}.}}
\end{figure}

\section{Time series}

\begin{figure}[H]
\begin{lstlisting}
xyplot(sunspot.year)
\end{lstlisting}
  \fcapside[\FBwidth] {\img{figs_lattice-crop}}
  {\caption*{\marginnote{\textbf{sunspot.year}. Yearly numbers of
        sunspots from 1700 to 1988.} A simple time-series is displayed
      as a lineplot, but taking care of arranging the $x$-axis for
      time measurements.}}
\end{figure}

\begin{figure}[H]
\begin{lstlisting}
xyplot(sunspot.year, aspect=.3, scales=list(y=list(rot=0)))
\end{lstlisting}
  \fcapside[\FBwidth] {\img{figs_lattice-crop}} {\caption*{A more
      comprehensive picture after aspect ratio has been lowered so as
      to better highlight the cyclic component.}}
\end{figure}

\begin{figure}[H]
\begin{lstlisting}
xyplot(sunspot.year, strip=FALSE, cut=list(number=2, overlap=.05))
\end{lstlisting}
  \fcapside[\FBwidth] {\img{figs_lattice-crop}}
  {\caption*{The same time series cut into two pieces with 5\% of overlap.}}
\end{figure}

\begin{figure}[H]
\begin{lstlisting}
xyplot(sunspot.year, strip=FALSE, strip.left=TRUE, 
       cut=list(number=2, overlap=.05))
\end{lstlisting}
  \fcapside[\FBwidth] {\img{figs_lattice-crop}}
  {\caption*{Now we highlight explicitely that the two series of
      measurements are related by adding a colored ribbon denoting
      time period.}}
\end{figure}

\begin{figure}[H]
\begin{lstlisting}
xyplot(zoo(discoveries))
\end{lstlisting}
  \fcapside[\FBwidth] {\img{figs_lattice-crop}}
  {\caption*{\marginnote{\textbf{discoveries}. The numbers of ``great''
        inventions and scientific discoveries in each year from 1860
        to 1959.} The \texttt{zoo} has to be loaded before using the
      above command. Briefly, it takes care of handling time-series
      data correctly, and it is interfaced to \texttt{lattice}'s
      \texttt{xyplot} as an S3 method (see \texttt{xyplot.zoo}).}}
\end{figure}


\chapter{Two-way graphics}

This chapter covers graphical displays for two-way relationships,
possibly by considering additional variables (numerical or
categorical) to highlight ternary relationships. Two-way graphics are
not limited to numerical variables as we may be interested in showing
the relationships between two ordered categorical variables, two
unordered or ``nominal'' variables. Moreover, as stated above,
categorical or discretized variables can be used to provide additional
information on top of a line- or scatter-plot by simply varying point
size, point or line colors, and so on. Of course, we could extend this
idea to the point of displaying sixth dimensions in a single graph
(e.g., using symbol with varying length and width, color, and shading
pattern). But, such a complex graph would likely be poorly
readable and uninformative in the end. So, this chapter basically
provides necessary \R command to create line-plot, scatter-plot,
bar-plot, dot-plot


\section{Lineplots}

% \begin{figure}[H]
% \begin{lstlisting}
% xyplot(uptake ~ conc, data=CO2, groups=Treatment)
% \end{lstlisting}
%   \fcapside[\FBwidth] {\img{figs_lattice-crop}}
%   {\caption*{\marginnote{\textbf{CO2}. The CO2 data frame has 84 rows
%         and 5 columns of data from an experiment on the cold tolerance
%         of the grass species \emph{Echinochloa crus-galli}.} Automatic
%     averaging}}
% \end{figure}

\begin{figure}[H]
\begin{lstlisting}
xyplot(uptake ~ conc, data=CO2, groups=Treatment, type="a")
\end{lstlisting}
  \fcapside[\FBwidth] {\img{figs_lattice-crop}}
  {\caption*{Automatic averaging.}}
\end{figure}




\section{Scatterplots}
The basic \R command for displaying a two-way scatterplot is
\texttt{xyplot}. A command like \texttt{xyplot(y ~ x)} will produce a
2D plot almost identical to what would be obtained using base
graphics, \texttt{plot(x, y)}. However, the default layout is
generally better and it looks more pretty.

\begin{figure}[H]
\begin{lstlisting}
xyplot(dist ~ speed, data=cars)
\end{lstlisting}
  \fcapside[\FBwidth] {\img{figs_lattice-crop}}
  {\caption*{\marginnote{\textbf{cars}. The data give the speed of
        cars and the distances taken to stop.  Note that the data were
        recorded in the 1920s.}This basic scatterplot show default
      options when calling the \texttt{xyplot} command. The formula
      interface is used to plot \texttt{dist} ($y$-axis) as a function
      of \texttt{speed} ($x$-axis), with automatic determination of
      axis units.}}
\end{figure}



\begin{figure}[H]
\begin{lstlisting}
xyplot(dist ~ speed, data=cars, pch=rbinom(nrow(cars), 1, .5)+1)
\end{lstlisting}
  \fcapside[\FBwidth] {\img{figs_lattice-crop}} {\caption*{We pick a
      random symbol (${\scriptstyle\Circle}=1$,
      ${\scriptstyle\triangle}=2$) for each observation, using the
      \texttt{pch=} argument. In fact, this argument is transferred to
      the \texttt{panel.xyplot} function that acts as the default
      panel function. Note that the vector of symbols should have the
      same length as the $x$ and $y$ components, otherwise recycling
      occurs.}}
\end{figure}


\begin{figure}[H]
\begin{lstlisting}
xyplot(dist ~ speed, data=cars, 
       col=with(cars, cut(speed, breaks=quantile(speed), 
                          include.lowest=TRUE)))
\end{lstlisting}
  \fcapside[\FBwidth] {\img{figs_lattice-crop}} {\caption*{Color
      (\texttt{col=}) of each observation depends of the quartile they
      belong to. Note that passing colors this way will override
      default theming options.}}
\end{figure}

\begin{figure}[H]
\begin{lstlisting}
xyplot(dist ~ speed, data=cars, pch=19,
       groups=with(cars, cut(speed, breaks=quantile(speed), 
                          include.lowest=TRUE)))
\end{lstlisting}
  \fcapside[\FBwidth] {\img{figs_lattice-crop}} {\caption*{This is
      basically the same code as previously shown except that we
      replaced the \texttt{col=} argument by \texttt{groups=}. This
      has the advantage of observing the current theme, and this will
      further facilitate the insertion of an automatic legend.}}
\end{figure}

\begin{figure}[H]
\begin{lstlisting}
xy <- as.data.frame(replicate(2, rnorm(1000)))
xyplot(V1 ~ V2, data=xy, pch=19, alpha=.5)
\end{lstlisting}
  \fcapside[\FBwidth] {\img{figs_lattice-crop}} {\caption*{When there
      are a high proportion of points that overlap, using transparent
      color may be useful. We replaced the default symbol with its
      filled counterpart. An equivalent way of specifying transparent
      color would be to use \texttt{rgb(.22, .49, .72, alpha=.5)}.}}
\end{figure}

\begin{figure}[H]
\begin{lstlisting}
dat <- data.frame(replicate(2, rnorm(500)), z=sample(0:40, 500, T))
cols <- colorRampPalette(brewer.pal(11, "RdBu"))(diff(range(dat$z)))
xyplot(X1 ~ X2, data=dat, col=cols[dat$z], pch=19, alpha=.5)
\end{lstlisting}
  \fcapside[\FBwidth] {\img{figs_lattice-crop}}
  {\caption*{Alpha-blending and color palette might be combined as
      well. Here, we used a pre-defined color scheme (Red to Blue)
      from the \texttt{RColorBrewer} package. As the selected palette
      has only 11 different colors, whereas the grouping factor,
      \texttt{z}, has 40 levels, we use linear interpolation to
      increase the number of available colors.}}
\end{figure}

\begin{figure}[H]
\begin{lstlisting}
xyplot(Sepal.Length ~ Petal.Length, data=iris, jitter.x=TRUE, 
       amount=.2)
\end{lstlisting}
  \fcapside[\FBwidth] {\img{figs_lattice-crop}}
  {\caption*{\marginnote{\textbf{iris}. This famous (Fisher's or
        Anderson's) iris data set gives the measurements in
        centimeters of the variables sepal length and width and petal
        length and width, respectively, for 50 flowers from each of 3
        species of iris.  The species are \emph{Iris setosa},
        \emph{versicolor}, and \emph{virginica}.}As an alternative to
      transparent colors, one may resort on ``jittering''. This is
      also useful when not so many points are available but show few
      variations on one dimension. The \texttt{panel.xyplot} function
      uses \texttt{jitter.x=} and \texttt{jitter.y=} to vary $x$ and
      $y$ coordinates by adding a random shift drawn from a uniform
      distribution, $\mathcal{U}(-a,a)$, where $a$ stands for the
      \texttt{amount=} parameter.}}
\end{figure}

% xyplot(uptake ~ conc, data=CO2, groups=Treatment)
% with(CO2, aggregate(uptake, list(conc=conc), mean))
% library(plyr)
% ddply(CO2, c("Treatment", "Plant"), transform, uptake.m = mean(uptake))

% xyplot(log(Volume) ~ log(Girth), data = trees)


% xyplot(FSC.H ~ SSC.H, gvhd10, subset=SSC.H < 1000, alpha=.1, cex=.6)
% xyplot(FSC.H ~ SSC.H, gvhd10, subset=SSC.H < 1000, panel = panel.hexbinplot)
% xyplot(FSC.H ~ SSC.H, gvhd10, subset=SSC.H < 1000, panel =
% panel.hexbinplot, trans=sqrt)


\begin{figure}[H]
\begin{lstlisting}
xyplot(dist ~ speed, data=cars,
       panel=function(x, y, ...) {
         panel.xyplot(x, y, ...)
         panel.rug(x, y, ...)
       })
\end{lstlisting}
\fcapside[\FBwidth]
{\img{figs_lattice-crop}}
{\caption*{It is possible to superimpose the univariate distribution
    of both series of measurement using ``rug'' plots. Usually, they
    remain quite discreet (read \emph{non-invasive}) but provide
    additional information to spot possible asymmetry. We need to ask
    explicitly for a custom \texttt{panel}, though.}}
\end{figure}

\begin{figure}[H]
\begin{lstlisting}
xyplot(log(Volume) ~ log(Girth), data=trees)
\end{lstlisting}
  \fcapside[\FBwidth] {\img{figs_lattice-crop}}
  {\caption*{\marginnote{\textbf{trees}. This data set provides
        measurements of the girth, height and volume of timber in 31
        felled black cherry trees.  Note that girth is the diameter of
        the tree (in inches) measured at 4 ft 6 in above the ground.}
      A simple log-log plot. Note that we would have to manually
      update the \texttt{scales=} component to provide more suitable
      annotations for the $x$ and $y$-axis.}}
\end{figure}

\begin{figure}[H]
\begin{lstlisting}
xyplot((1:200)/20 ~ (1:200)/20, type=c("p", "g"),
       scales=list(x=list(log=10), y=list(log=10)),
       xscale.components=xscale.components.log10.3,
       yscale.components=yscale.components.log10.3)
\end{lstlisting}
  \fcapside[\FBwidth] {\img{figs_lattice-crop}} {\caption*{Instead of
      transforming variables in the formula, it is easier and safer to
      do this through the \texttt{scales=} parameter. The
      \texttt{[x|y]scale.components} are convenient functions that
      help to annotate axes with correct units and tick marks
      spacing.}}
\end{figure}


\begin{figure}[H]
\begin{lstlisting}
xyplot(dist ~ speed, data=cars, scales=list(tck=c(1,0)))
\end{lstlisting}
  \fcapside[\FBwidth] {\img{figs_lattice-crop}} {\caption*{To get ride
      of ticks on opposite axes, we can change default values for
      \texttt{tck=} in the \texttt{scales=} component. The
      \texttt{tck=} parameter controls the length of tick marks;
      however, with a vector of length 2 it can be used to deal with
      left/bottom and right/top axis separately.}}
\end{figure}

\begin{figure}[H]
\begin{lstlisting}
xyplot(dist ~ speed, data=cars, scales=list(alternating=3))
\end{lstlisting}
  \fcapside[\FBwidth] {\img{figs_lattice-crop}} {\caption*{To annotate
      both axes, we can alter the \texttt{alternating=} parameter. In
      most case, however, adding grid lines in the background should
      provide enough information. Using \texttt{alternating=2} would
      reverse the annotation of axis (right/top instead of
      left/bottom).}}
\end{figure}

\chapter{Multi-way graphics}

This chapter focus on multi-variable displays, where usually two-way
graphics are conditioned on values taken by one or more variables, or
a combination thereof. These so-called ``\index{treillis} displays'' are very
good at conveying information about trend or variation between two
numerical variables across the levels of a third factor.



\chapter{Customizing theme and panels}

\appendix
\backmatter

\printindex

\end{document}
